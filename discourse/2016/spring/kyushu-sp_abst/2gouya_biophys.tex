\documentclass[12pt]{jsarticle}
\usepackage[top=30truemm,bottom=30truemm,left=25truemm,right=25truemm]{geometry}
\usepackage{fancyhdr}
\pagestyle{fancy}
\renewcommand{\headrulewidth}{0pt}
\renewcommand{\footrulewidth}{0pt}
\lhead{}
\rhead{}
\chead{}
\lfoot{}
\rfoot{\fontsize{12pt}{0pt}\selectfont
九州春の談話会}% ここに談話会名を入力してください。
\cfoot{\thepage}
\renewcommand{\labelitemi}{$\bigcirc$}
\makeatletter
\makeatother 
%
%
%========講演者の方はここより下の部分の必要事項を入力してください。==============
%
%
\title{生物の動きと物理}% ここに講演タイトルを入力してください。
\author{合屋 純\\% ここに名前を入力してください。
{\small
九州大学理学部物理学科4年 % ここに大学学部学科学年を入力してください。
}}
\date{2016年6月25日}% 談話会の日付を入力してください。
\begin{document}
\maketitle
\thispagestyle{fancy}
\section{はじめに}
% 本文を入力してください。
「生物と無生物との違いはなんですか?」という問いに、私たちはどのように答えられるでしょうか。この問に対しての答えは「生物は生殖活動を行う」や「生物は自ら決定・行動する」など、様々な回答がされると思います。しかし、より明確な境界というものは非常に難しく思われます。一人の人間から、脳などの臓器、細胞、タンパク質、遺伝子とスケールを少しずつ落としていく時、どこからが無生物なのでしょうか。これに対して物理学から生物にアプローチすることで、生物というものを深く知り、生物モデルに定量的な理解を与えます。物理学を用いて各階層の生命現象を理解し、この階層をつなぐ原理や原則を見出すことによって生命現象を解き明かしていく試みが、多くの物理学者、そして生物学者によって行われています。
\section{講演内容}
% 本文を入力してください。
物理学がどのようにして生物を取り入れているか、具体的な例を示しながら紹介を行います。その中でも特に、生物の動きに着目したモデルについて中心的に説明します。また、題材は生物についてではありますが生物学の予備知識は不要な内容にします。取り扱う現象についても基本的に簡単なもの(例:イワシの集団による渦運動)であり、説明を前提としますので、これまでに生物学を学んだことがない方にも難しくない講演を行います。物理、数学の知識としても最低限として、大学入学当時に学ぶ偏微分などを知っていれば取り組みやすいという程度です。
\begin{thebibliography}{9}% 参考文献を入力してください。
\bibitem{}E. シュレーディンガー, 生命とは何か, 岩波書店 (2008)
\bibitem{}西浦 廉政, パターン形成の数理, 岩波書店 (1999)
\bibitem{}Yusuke T. Maeda, Junya Inose, Miki Y. Matsuo, Suguru Iwaya, Masaki Sano (2008) Ordered Patterns of Cell Shape and Orientational Correlation during Spontaneous Cell Migration
\bibitem{}Vicsek, T., Czirók, A., Ben-Jacob, E., Cohen, I., Shochet, O (1995) Novel Type of Phase Transition in a System of Self-Driven Particles. 
\bibitem{}Ziane Izri, Marjolein N. van der Linden, S´ebastien Michelin, and Olivier Dauchot (1994) Self-propulsion of pure water droplets by spontaneous Marangoni stress driven motion 
\end{thebibliography}
\end{document}